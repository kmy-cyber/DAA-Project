\documentclass[12pt]{article}
\usepackage[utf8]{inputenc}
\usepackage[spanish]{babel}
\usepackage{geometry}
\usepackage{graphicx}
\usepackage{hyperref}
\usepackage{listings}
\usepackage{xcolor}
\usepackage{amsmath}
\usepackage{amssymb}

\geometry{a4paper, left=2.5cm, right=2.5cm, top=2.5cm, bottom=2.5cm}

\definecolor{codegreen}{rgb}{0,0.6,0}
\definecolor{codegray}{rgb}{0.5,0.5,0.5}
\definecolor{codepurple}{rgb}{0.58,0,0.82}
\definecolor{backcolour}{rgb}{0.95,0.95,0.92}

\lstdefinestyle{mystyle}{
    backgroundcolor=\color{backcolour},   
    commentstyle=\color{codegreen},
    keywordstyle=\color{magenta},
    numberstyle=\tiny\color{codegray},
    stringstyle=\color{codepurple},
    basicstyle=\ttfamily\footnotesize,
    breakatwhitespace=false,         
    breaklines=true,                 
    captionpos=b,                    
    keepspaces=true,                 
    numbers=left,                    
    numbersep=5pt,                  
    showspaces=false,                
    showstringspaces=false,
    showtabs=false,                  
    tabsize=2
}
\lstset{style=mystyle}

\begin{document}

\begin{titlepage}
    \centering

    \vspace*{-1cm}
    \includegraphics[width=4cm]{matcom.jpeg}
    \vspace{1cm}

    {\Large \textbf{Universidad de La Habana}}\\[0.3cm]
    {\large \textbf{Facultad de Matemática y Computación}}\\[1cm]

    {\large Asignatura: Diseño y Análisis de Algoritmos}\\[2cm]

    {\LARGE \textbf{Conectando la UH}}\\[0.8cm]

    {\Large \textbf{Integrantes}}\\[0.6cm]
    {\large Jabel Resendiz Aguirre}\\
    {\large Noel Pérez Calvo}\\
    {\large Arianne Camila Palancar Ochando }\\[1.7cm]

    {\large Carrera: Ciencia de la Computación}\\[2cm]

    \vfill
    {\large \today}
\end{titlepage}


\tableofcontents
\newpage

\section{Fase 1: Formalización del Problema}

\subsection{Descripción Formal del Problema}
A partir de la descripción informal de ``Conectando la UH'', el objetivo de esta fase es
traducir el problema a un modelo matemático preciso y libre de ambigüedades.
El modelo debe especificar:

\begin{itemize}
    \item la estructura de los datos de entrada,
    \item las restricciones que deben cumplirse en la solución,
    \item y la forma en que se caracteriza una solución óptima.
\end{itemize}

El problema consiste, en esencia, en determinar una configuración óptima de conexiones dentro de la Universidad de La Habana, cumpliendo condiciones específicas de conectividad, costo y estructura.

\subsection{Estructuras de Datos de Entrada}
Para formalizar el problema, se definen las siguientes estructuras de entrada:

\begin{itemize}
    \item Un conjunto de nodos:
    \[
        V = \{v_1, v_2, \ldots, v_n\}
    \]
    representando facultades, edificios o puntos estratégicos.

    \item Un conjunto de posibles conexiones (aristas):
    \[
        E = \{(u, v) \mid u, v \in V, u \neq v\}
    \]
    donde cada conexión tiene asociado un costo:
    \[
        c : E \rightarrow \mathbb{R}_{>0}
    \]

    \item Posibles restricciones de conectividad predefinidas, por ejemplo:
    \[
        R \subseteq E
    \]
    que representan enlaces obligatorios o prohibidos.

    \item Un parámetro que determina el tipo de estructura deseada:
    \[
        \text{tipo} \in \{\text{árbol}, \text{grafo acotado}, \text{red redundante}\}
    \]
\end{itemize}

\subsection{Formalización Matemática}
Sea $G = (V, E)$ el grafo que representa las posibles conexiones entre puntos de la UH.
La solución del problema consiste en encontrar un subgrafo:

\[
    G' = (V', E') \quad\text{tal que}\quad V' \subseteq V,\; E' \subseteq E
\]

que cumpla las restricciones del problema y optimice una función objetivo dada.

\subsection{Restricciones del Problema}

Las restricciones pueden incluir:

\begin{enumerate}
    \item \textbf{Conectividad global:}
    \[
        G' \text{ es conexo}
    \]

    \item \textbf{Respeto a enlaces obligatorios:}
    \[
        R_{\text{oblig}} \subseteq E'
    \]

    \item \textbf{Prohibición de ciertos enlaces:}
    \[
        R_{\text{prohib}} \cap E' = \emptyset
    \]

    \item \textbf{Límites estructurales:}  
    Por ejemplo, si se desea que la solución sea un árbol:
    \[
        |E'| = |V'| - 1 \quad \wedge \quad G' \text{ es acíclico}
    \]

    \item \textbf{Presupuesto máximo:}
    \[
        \sum_{e \in E'} c(e) \leq B
    \]
\end{enumerate}

\subsection{Propiedades de una Solución Válida}
Una solución es válida si:

\begin{itemize}
    \item cumple todas las restricciones estructurales,  
    \item es conexa (o satisface la conectividad solicitada),  
    \item respeta los enlaces obligatorios y prohibidos,
    \item se mantiene dentro del presupuesto si aplica.
\end{itemize}

\subsection{Función Objetivo}
Dependiendo del diseño seleccionado, la función objetivo puede ser:

\begin{itemize}
    \item \textbf{Minimizar el costo total de instalación}:
    \[
        \min \sum_{e \in E'} c(e)
    \]

    \item \textbf{Maximizar la redundancia}:
    \[
        \max |E'| \quad\text{bajo restricciones de costo}
    \]

    \item \textbf{Minimizar la distancia máxima entre nodos} (problema del árbol generador de diámetro mínimo):
    \[
        \min \max_{u, v \in V'} d_{G'}(u, v)
    \]
\end{itemize}

La función objetivo específica será determinada por el diseño final deseado en el contexto de ``Conectando la UH''.

\section{Fase 2: Análisis de Complejidad Computacional}

\section{Fase 3:  Diseño de Soluciones Algorítmicas}

\section{Fase 4: Implementación y Análisis Experimental}

\end{document}
