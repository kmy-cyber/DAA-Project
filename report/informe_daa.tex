\documentclass[12pt]{article}
\usepackage[utf8]{inputenc}
\usepackage[spanish]{babel}
\usepackage{geometry}
\usepackage{graphicx}
\usepackage{hyperref}
\usepackage{listings}
\usepackage{xcolor}
\usepackage{amsmath}
\usepackage{amssymb}

\geometry{a4paper, left=2.5cm, right=2.5cm, top=2.5cm, bottom=2.5cm}

\definecolor{codegreen}{rgb}{0,0.6,0}
\definecolor{codegray}{rgb}{0.5,0.5,0.5}
\definecolor{codepurple}{rgb}{0.58,0,0.82}
\definecolor{backcolour}{rgb}{0.95,0.95,0.92}

\lstdefinestyle{mystyle}{
    backgroundcolor=\color{backcolour},   
    commentstyle=\color{codegreen},
    keywordstyle=\color{magenta},
    numberstyle=\tiny\color{codegray},
    stringstyle=\color{codepurple},
    basicstyle=\ttfamily\footnotesize,
    breakatwhitespace=false,         
    breaklines=true,                 
    captionpos=b,                    
    keepspaces=true,                 
    numbers=left,                    
    numbersep=5pt,                  
    showspaces=false,                
    showstringspaces=false,
    showtabs=false,                  
    tabsize=2
}
\lstset{style=mystyle}

\begin{document}

\begin{titlepage}
    \centering

    \vspace*{-1cm}
    \includegraphics[width=4cm]{matcom.jpeg}
    \vspace{1cm}

    {\Large \textbf{Universidad de La Habana}}\\[0.3cm]
    {\large \textbf{Facultad de Matemática y Computación}}\\[1cm]

    {\large Asignatura: Diseño y Análisis de Algoritmos}\\[2cm]

    {\LARGE \textbf{Conectando la UH}}\\[0.8cm]

    {\Large \textbf{Integrantes}}\\[0.6cm]
    {\large Jabel Resendiz Aguirre}\\
    {\large Noel Pérez Calvo}\\
    {\large Arianne Camila Palancar Ochando }\\[1.7cm]

    {\large Carrera: Ciencia de la Computación}\\[2cm]

    \vfill
    {\large \today}
\end{titlepage}


\tableofcontents
\newpage

\section{Fase 1: Formalización del Problema}

\subsection{Descripción Formal del Problema}

El objetivo es diseñar una red de fibra óptica que conecte todos los edificios 
principales de la Universidad de La Habana mediante enlaces posibles provistos por ETECSA. 
Cada enlace tiene un costo de instalación y cada edificio posee un límite máximo 
de puertos disponibles. Se requiere encontrar una configuración de conexiones que:

\begin{itemize}
	\item conecte todos los edificios,
	\item respete los límites de puertos por edificio,
	\item minimice el costo total de instalación.
\end{itemize}

El problema se formaliza como una variante acotada del \textit{árbol generador mínimo}.

\subsection{Estructuras de Datos de Entrada}

Se define la entrada mediante las siguientes estructuras:

\begin{itemize}
	\item Un conjunto de edificios:
	\[
	V = \{v_1, v_2, \ldots, v_n\}.
	\]
	
	\item Un conjunto de posibles enlaces de fibra óptica:
	\[
	E \subseteq \{\{u,v\} \mid u,v \in V,\; u \neq v\}.
	\]
	
	\item Un costo de instalación para cada enlace:
	\[
	c : E \rightarrow \mathbb{R}_{>0}.
	\]
	
	\item Un límite de puertos (grado máximo permitido) en cada edificio:
	\[
	d : V \rightarrow \mathbb{Z}_{\ge 1}.
	\]
\end{itemize}

\subsection{Formalización Matemática}

Sea \(G=(V,E)\) el grafo de infraestructura disponible.  
La solución buscada es un subconjunto de aristas:
\[
T \subseteq E
\]
que define un subgrafo \(G_T = (V, T)\).

\subsection{Restricciones del Problema}

El subgrafo \(G_T\) debe satisfacer:

\begin{enumerate}
	\item \textbf{Conectividad:}
	\[
	G_T \text{ es conexo}.
	\]
	
	\item \textbf{Estructura de árbol:}
	\[
	|T| = |V| - 1.
	\]
	
	\item \textbf{Límite de puertos por edificio:}
	\[
	\deg_{G_T}(v) \le d(v), \qquad \forall v \in V.
	\]
\end{enumerate}

\subsection{Función Objetivo}

Minimizar el costo total de instalación:

\[
\min_{T \subseteq E} \sum_{e \in T} c(e)
\]

sujeto a las restricciones anteriores.

\section{Fase 2: Análisis de Complejidad Computacional}

En esta fase se determina la dificultad computacional del problema formalizado en la Fase~1.
Demostraremos que la versión de decisión del problema es NP-completa, y por lo tanto, 
la versión de optimización es NP-dura.

\subsection{Versión de Decisión del Problema}

Dado un grafo $G=(V,E)$, costos $c(e)$, límites de grado $d(v)$ y un valor $K$, 
definimos la siguiente pregunta:

\[
\text{¿Existe un subconjunto } T \subseteq E \text{ tal que }
\begin{cases}
	G_T = (V,T) \text{ es un árbol}, \\
	\deg_{G_T}(v) \le d(v) \;\; \forall v \in V, \\
	\sum_{e \in T} c(e) \le K?
\end{cases}
\]

\subsection{Pertenencia a NP}

Dado un conjunto de aristas $T$, quiere verificarse que se puede comprobar en tiempo polinomial, 
lo cual se cumplirá si se cumplen las siguientes condiciones:

\begin{itemize}
	\item $G_T$ es conexo (mediante un recorrido BFS/DFS),
	\item $|T| = |V|-1$, es decir el subgrafo $G_T$ forma un árbol,
	\item $\deg_{G_T}(v) \le d(v)$ para todo $v \in V$,
	\item $\sum_{e \in T} c(e) \le K$.
\end{itemize}

Por lo tanto, el problema pertenece a NP.

\subsection{Demostración de NP-completitud}

Para demostrar que dicho problema es NP-completo, se presenta una 
reducción en tiempo polinomial desde el problema \textsc{Hamiltonian Path} en 
grafos no dirigidos, el cual es NP-completo.

\subsubsection{Problema de Partida: Hamiltonian Path}

Dado un grafo no dirigido $G=(V,E)$, el problema \textsc{Hamiltonian Path} 
pregunta si existe un camino que visite todos los vértices exactamente una vez.

\subsubsection{Construcción de la Reducción}

A partir de una instancia de \textsc{Hamiltonian Path}, construimos una instancia 
del problema de la siguiente manera:

\begin{itemize}
	\item Se toma el mismo grafo $G=(V,E)$.
	\item Para cada arista $e \in E$, se asigna un costo $c(e)=1$.
	\item Se fija un límite de grado uniforme:
	\[
	d(v) = 2 \quad \forall v \in V.
	\]
	\item Se establece el umbral de costo:
	\[
	K = |V| - 1.
	\]
\end{itemize}

La construcción es claramente polinomial.

\subsubsection{Correctitud de la Reducción}

\paragraph{($\Rightarrow$)}  
Si $G$ posee un camino hamiltoniano, dicho camino contiene $|V|-1$ aristas, 
es conexo, acíclico y cada vértice tiene grado a lo sumo $2$.  
Por lo tanto, constituye un conjunto $T$ válido para el problema
con costo total $|V|-1 \le K$.

\paragraph{($\Leftarrow$)}  
Si existe un conjunto $T$ que satisface las restricciones del problema,
entonces $T$ es un árbol con $|V|-1$ aristas y $\deg_{G_T}(v) \le 2$ para todo $v$.  
La única estructura de árbol donde todos los grados son a lo sumo $2$ es un camino.  
Por lo tanto, $T$ es un camino hamiltoniano del grafo original.

\subsubsection{Conclusión}

La existencia de un $T$ válido para \textsc{DC-MST-Decision} es equivalente 
a la existencia de un camino hamiltoniano en $G$.  
Dado que la reducción es polinomial y \textsc{Hamiltonian Path} es NP-completo, se concluye:

\[
\textsc{DC-MST-Decision} \text{ es NP-completo}.
\]

\subsection{Consecuencia}

Dado que la versión de decisión es NP-completa, la versión de optimización 
(\textit{Degree-Constrained Minimum Spanning Tree}) es NP-dura:

\[
\text{DC-MST es NP-dura}.
\]

Esto implica que no se conoce un algoritmo polinomial que resuelva el problema de forma
óptima para instancias generales, salvo que $\mathrm{P} = \mathrm{NP}$.

\section{Fase 3:  Diseño de Soluciones Algorítmicas}

\section{Fase 4: Implementación y Análisis Experimental}

\end{document}
