\documentclass[12pt]{article}
\usepackage[utf8]{inputenc}
\usepackage[spanish]{babel}
\usepackage{geometry}
\usepackage{graphicx}
\usepackage{hyperref}
\usepackage{listings}
\usepackage{xcolor}
\usepackage{amsmath}
\usepackage{amssymb}

\geometry{a4paper, left=2.5cm, right=2.5cm, top=2.5cm, bottom=2.5cm}

\definecolor{codegreen}{rgb}{0,0.6,0}
\definecolor{codegray}{rgb}{0.5,0.5,0.5}
\definecolor{codepurple}{rgb}{0.58,0,0.82}
\definecolor{backcolour}{rgb}{0.95,0.95,0.92}

\lstdefinestyle{mystyle}{
    backgroundcolor=\color{backcolour},   
    commentstyle=\color{codegreen},
    keywordstyle=\color{magenta},
    numberstyle=\tiny\color{codegray},
    stringstyle=\color{codepurple},
    basicstyle=\ttfamily\footnotesize,
    breakatwhitespace=false,         
    breaklines=true,                 
    captionpos=b,                    
    keepspaces=true,                 
    numbers=left,                    
    numbersep=5pt,                  
    showspaces=false,                
    showstringspaces=false,
    showtabs=false,                  
    tabsize=2
}
\lstset{style=mystyle}

\begin{document}

\begin{titlepage}
    \centering

    \vspace*{-1cm}
    \includegraphics[width=4cm]{matcom.jpeg}
    \vspace{1cm}

    {\Large \textbf{Universidad de La Habana}}\\[0.3cm]
    {\large \textbf{Facultad de Matemática y Computación}}\\[1cm]

    {\large Asignatura: Diseño y Análisis de Algoritmos}\\[2cm]

    {\LARGE \textbf{Conectando la UH}}\\[0.8cm]

    {\Large \textbf{Integrantes}}\\[0.6cm]
    {\large Jabel Resendiz Aguirre}\\
    {\large Noel Pérez Calvo}\\
    {\large Arianne Camila Palancar Ochando }\\[1.7cm]

    {\large Carrera: Ciencia de la Computación}\\[2cm]

    \vfill
    {\large \today}
\end{titlepage}


\tableofcontents
\newpage

\begin{center}
\LARGE \textbf{Problema}
\end{center}
\addcontentsline{toc}{section}{Problema}

La Universidad de La Habana, en su constante búsqueda de la excelencia académica y la innovación, 
se ha embarcado en un proyecto crucial para modernizar y expandir su infraestructura de red. Nuestro 
objetivo es dotar a todas nuestras facultades, centros de investigación y edificios administrativos con
conectividad de fibra óptica de alta velocidad. Para este fin, contamos con el valioso apoyo técnico y
logístico de ETECSA (Empresa de Telecomunicaciones de Cuba S.A.).

Nos enfrentamos a un desafío de diseño de red que requiere una solución óptima. Necesitamos 
interconectar todos los edificios principales de la UH con fibra óptica, creando una red robusta 
y eficiente. Cada posible conexión de fibra entre dos edificios tiene un costo de instalación asociado,
que incluye desde los permisos internos y la mano de obra especializada de ETECSA hasta los materiales
y las obras civiles necesarias.

Sin embargo, ETECSA ha establecido una restricción técnica fundamental que debemos respetar:  

En cada edificio, la conexión de la fibra óptica se gestionará a través de un equipo de red central 
(un router o switch principal) que ellos nos proporcionan. Estos equipos tienen una capacidad limitada 
de puertos. Esto significa que un equipo en un edificio específico solo puede manejar un número máximo 
de conexiones de fibra óptica directas a otros edificios. Exceder este límite implicaría la necesidad 
de instalar equipos adicionales mucho más caros y complejos, o la implementación de soluciones de red 
alternativas que ETECSA no puede garantizar o que dispararían drásticamente el presupuesto del proyecto.

Nuestro objetivo principal es diseñar la red de fibra óptica que conecte todos nuestros edificios 
principales de la manera más económica posible. Esto implica seleccionar las rutas de fibra de tal 
forma que:

\begin{enumerate}
	\item Todos los edificios estén interconectados a la red principal de la universidad, sin crear bucles 
innecesarios (buscamos una estructura de red eficiente).
	\item Ningún equipo de red en ningún edificio exceda su capacidad máxima de conexiones directas 
(es decir, el número de cables de fibra que llegan o salen de un edificio no puede superar el límite 
de puertos del equipo de ETECSA).
	\item El costo total de instalación de toda la red sea el mínimo posible.
\end{enumerate}

Una planificación subóptima podría resultar en un sobrecosto significativo para la universidad, la 
necesidad de adquirir hardware de red adicional no previsto, o en una red ineficiente que no cumpla 
con las especificaciones técnicas y presupuestarias acordadas con ETECSA.

\clearpage
\begin{center}
\LARGE \textbf{Formalización del Problema}
\end{center}
\addcontentsline{toc}{section}{Fase 1: Formalización del Problema}

El objetivo es diseñar una red de fibra óptica que conecte todos los edificios 
principales de la Universidad de La Habana mediante enlaces posibles provistos por ETECSA. 
Cada enlace tiene un costo de instalación y cada edificio posee un límite máximo 
de puertos disponibles. Se requiere encontrar una configuración de conexiones que:

\begin{itemize}
	\item conecte todos los edificios,
	\item respete los límites de puertos por edificio,
	\item minimice el costo total de instalación.
\end{itemize}

\subsection*{Notaci\'on}
\addcontentsline{toc}{subsection}{Notaci\'on}

En el lenguaje matemático, podemos definir la estructura del problema como un grafo simple y no dirigido $G=(V,E)$, donde:

\begin{itemize}
	\item Un conjunto de edificios:
	\[
	V = \{v_1, v_2, \ldots, v_n\}.
	\]
	
	\item Un conjunto de posibles enlaces de fibra óptica: $E=(e_1,e_2,...,e_m)$
	\[
	E \subseteq \{\{u,v\} \mid u,v \in V,\; u \neq v\}.
	\]
	
	\item Un costo de instalación para cada enlace:
	\[
	c : E \rightarrow \mathbb{R}_{>0}.
	\]
	
	\item Un límite de puertos (grado máximo permitido) en cada edificio:
	\[
	d : V \rightarrow \mathbb{Z}_{\ge 1}.
	\]
\end{itemize}

\subsection*{Restricciones del Problema}
\addcontentsline{toc}{subsection}{Restricciones del Problema}

Sea un subgrafo de G como \(T=(V^T,E^T)\) debe satisfacer que:

\begin{enumerate}

	\item \textbf{Conectividad:}
	\[
	T \text{ es conexo } \rightarrow V^T = V
	\]
	
	\item \textbf{Estructura de árbol:}
	\[
	T \text{ no contiene ciclos} \rightarrow |E^T| = |V| - 1.
	\]
	
	\item \textbf{Límite de puertos por edificio:}
	\[
	\deg_{T}(v) \le d(v), \qquad \forall v \in V.
	\]
\end{enumerate}

donde se define $deg_{T}(v)$ como el grado del v\'ertice $v$ (tambi\'en llamado cardinalidad de su vecindad) en el grafo T.
Se desea conseguir el subgrafo $T$ tal que minimiza el costo de la suma de los pesos de sus aristas, es decir:

\begin{enumerate}
	\item \textbf{Funci\'on objetivo:}
	\[
	\min_{T \subseteq E} \sum_{e \in T} c(e)
	\]
\end{enumerate}

sujeto a las restricciones anteriores.

\newpage
\begin{center}
\LARGE \textbf{Análisis de Complejidad Computacional}
\end{center}
\addcontentsline{toc}{section}{Fase 2: Análisis de Complejidad Computacional}

En esta fase se determina la dificultad computacional del problema formalizado en la Fase~1.
Demostraremos que la versión de decisión del problema es NP-completa, y por lo tanto, 
la versión de optimización es NP-dura.

\subsection*{Versión de Decisión del Problema}

Dado un grafo $G=(V,E)$, costos $c(e)$, límites de grado $d(v)$ y un valor $K$, 
definimos la siguiente pregunta:

\[
\text{¿Existe un subconjunto } T \subseteq E \text{ tal que }
\begin{cases}
	G_T = (V,T) \text{ es un árbol}, \\
	\deg_{G_T}(v) \le d(v) \;\; \forall v \in V, \\
	\sum_{e \in T} c(e) \le K?
\end{cases}
\]

\subsection*{Pertenencia a NP}

Dado un conjunto de aristas $T$, quiere verificarse que se puede comprobar en tiempo polinomial, 
lo cual se cumplirá si se cumplen las siguientes condiciones:

\begin{itemize}
	\item $G_T$ es conexo (mediante un recorrido BFS/DFS),
	\item $|T| = |V|-1$, es decir el subgrafo $G_T$ forma un árbol,
	\item $\deg_{G_T}(v) \le d(v)$ para todo $v \in V$,
	\item $\sum_{e \in T} c(e) \le K$.
\end{itemize}

Por lo tanto, el problema pertenece a NP.

\subsection*{Demostración de NP-completitud}

Para demostrar que dicho problema es NP-completo, se presenta una 
reducción en tiempo polinomial desde el problema \textsc{Hamiltonian Path} en 
grafos no dirigidos, el cual es NP-completo.

\subsubsection*{Problema de Partida: Hamiltonian Path}

Dado un grafo no dirigido $G=(V,E)$, el problema \textsc{Hamiltonian Path} 
pregunta si existe un camino que visite todos los vértices exactamente una vez.

\subsubsection*{Construcción de la Reducción}

A partir de una instancia de \textsc{Hamiltonian Path}, construimos una instancia 
del problema de la siguiente manera:

\begin{itemize}
	\item Se toma el mismo grafo $G=(V,E)$.
	\item Para cada arista $e \in E$, se asigna un costo $c(e)=1$.
	\item Se fija un límite de grado uniforme:
	\[
	d(v) = 2 \quad \forall v \in V.
	\]
	\item Se establece el umbral de costo:
	\[
	K = |V| - 1.
	\]
\end{itemize}

La construcción es claramente polinomial.

\subsubsection*{Correctitud de la Reducción}

\paragraph{($\Rightarrow$)}  
Si $G$ posee un camino hamiltoniano, dicho camino contiene $|V|-1$ aristas, 
es conexo, acíclico y cada vértice tiene grado a lo sumo $2$.  
Por lo tanto, constituye un conjunto $T$ válido para el problema
con costo total $|V|-1 \le K$.

\paragraph{($\Leftarrow$)}  
Si existe un conjunto $T$ que satisface las restricciones del problema,
entonces $T$ es un árbol con $|V|-1$ aristas y $\deg_{G_T}(v) \le 2$ para todo $v$.  
La única estructura de árbol donde todos los grados son a lo sumo $2$ es un camino.  
Por lo tanto, $T$ es un camino hamiltoniano del grafo original.

\subsubsection*{Conclusión}

La existencia de un $T$ válido para \textsc{DC-MST-Decision} es equivalente 
a la existencia de un camino hamiltoniano en $G$.  
Dado que la reducción es polinomial y \textsc{Hamiltonian Path} es NP-completo, se concluye:

\[
\textsc{DC-MST-Decision} \text{ es NP-completo}.
\]

\subsection*{Consecuencia}

Dado que la versión de decisión es NP-completa, la versión de optimización 
(\textit{Degree-Constrained Minimum Spanning Tree}) es NP-dura:

\[
\text{DC-MST es NP-dura}.
\]

Esto implica que no se conoce un algoritmo polinomial que resuelva el problema de forma
óptima para instancias generales, salvo que $\mathrm{P} = \mathrm{NP}$.

\newpage
\begin{center}
\LARGE \textbf{Diseño de Soluciones Algorítmicas}
\end{center}
\addcontentsline{toc}{section}{Fase 3: Diseño de Soluciones Algorítmicas}

El problema de construir un árbol expansivo con restricciones de grado
(DC-MST) es NP-difícil, lo que implica que no existe un algoritmo
polinomial conocido capaz de garantizar la solución óptima para
instancias generales. Por ello, en esta sección se exploran distintas
estrategias algorítmicas para abordar el problema, combinando
métodos exactos y heurísticos.

Presentaremos los algoritmos propuestos de manera estructurada,
incluyendo:

\begin{itemize}
    \item Una descripción de la idea principal de cada algoritmo.
    \item El pseudocódigo correspondiente para su implementación.
    \item El análisis de complejidad computacional.
\end{itemize}

Esto permitirá evaluar las ventajas y limitaciones de cada enfoque,
y proporcionará una base sólida para la implementación experimental
y la comparación de resultados en la Fase 4.

\subsection*{Algoritmo de Fuerza Bruta: Exploración Exhaustiva del Espacio de Árboles Expansivos}
\addcontentsline{toc}{subsection}{Algoritmo de Fuerza Bruta: Exploración Exhaustiva del Espacio de Árboles Expansivos}

Dado que el problema de encontrar un árbol expansivo de costo mínimo con
restricciones de grado es NP-difícil, una primera aproximación consiste en
analizar exhaustivamente todas las posibles soluciones y seleccionar aquella
que cumple las restricciones y minimiza el costo total.

Un candidato a solución válida debe ser un \textbf{árbol expansivo}, es decir,
un subgrafo conexo y acíclico que contiene todos los vértices del grafo y
exactamente $|V|-1$ aristas. El algoritmo de fuerza bruta consiste en enumerar
todos los subconjuntos de aristas del grafo y verificar cuáles de ellos cumplen
las condiciones necesarias para ser un árbol expansivo factible.

\subsubsection*{Idea del Algoritmo}

Sea $G=(V,E)$ un grafo no dirigido con $|V|=n$ vértices y $|E|=m$ aristas.
El procedimiento seguido es el siguiente:

\begin{enumerate}
    \item Enumerar todos los subconjuntos $S \subseteq E$.
    \item Descartar aquellos subconjuntos que no contengan exactamente $n-1$ aristas.
    \item Para cada subconjunto candidato:
    \begin{itemize}
        \item Verificar que el subgrafo inducido sea conexo y acíclico, utilizando
        una estructura de datos \emph{Disjoint Set Union} (DSU).
        \item Comprobar que el grado de cada vértice no exceda su límite máximo
        permitido.
        \item Calcular el costo total de las aristas seleccionadas.
    \end{itemize}
    \item Conservar el subconjunto válido cuyo costo total sea mínimo.
\end{enumerate}

Este algoritmo garantiza encontrar la solución óptima, aunque a costa de un
tiempo de ejecución exponencial.

\subsubsection*{Pseudocódigo}

\begin{lstlisting}[language=]
BruteForce-DCMST(G):
    bestCost <- +infinity
    bestTree <- empty

    for each subset S of E:
        if |S| != |V| - 1:
            continue

        initialize DSU with |V| elements
        deg[v] <- 0 for all v in V
        cost <- 0
        valid <- true

        for each edge (u,v) in S:
            if deg[u] + 1 > d(u) or deg[v] + 1 > d(v):
                valid <- false
                break

            deg[u] <- deg[u] + 1
            deg[v] <- deg[v] + 1

            if DSU.find(u) == DSU.find(v):
                valid <- false
                break

            DSU.union(u,v)
            cost <- cost + c(u,v)

        if valid and cost < bestCost:
            bestCost <- cost
            bestTree <- S

    return bestCost, bestTree
\end{lstlisting}

\subsubsection*{Análisis de Complejidad}

Sea $n = |V|$ el número de vértices y $m = |E|$ el número de aristas del grafo.

\begin{itemize}
    \item El algoritmo enumera todos los subconjuntos posibles de aristas, lo cual
    implica $2^m$ iteraciones.
    \item Para cada subconjunto candidato con $n-1$ aristas, se realizan:
    \begin{itemize}
        \item Operaciones de unión y búsqueda en la estructura DSU, con costo
        $O(n \alpha(n)) \approx O(n)$.
        \item Verificación de los límites de grado y cálculo del costo total,
        ambos en $O(n)$.
    \end{itemize}
\end{itemize}

Por tanto, la complejidad temporal total del algoritmo es:

\[
\boxed{
O(2^m \cdot n)
}
\]

En el peor caso, cuando el grafo es denso y $m = O(n^2)$, la complejidad crece
como $O(2^{n^2})$, lo cual hace que este enfoque sea impracticable para
instancias de tamaño moderado o grande.

\subsubsection*{Conclusión}

El algoritmo de fuerza bruta permite obtener la solución óptima del problema de
diseño de la red de fibra óptica bajo restricciones de grado. Sin embargo, su
costo computacional exponencial limita su uso a instancias muy pequeñas. Esto
justifica la necesidad de desarrollar algoritmos heurísticos o aproximados,
los cuales se abordan en las siguientes secciones.

\subsection*{Algoritmo Heurístico: Reducción del Árbol de Oportunidades}
\addcontentsline{toc}{subsection}{Algoritmo Heurístico: Reducción del Árbol de Oportunidades}

Con el objetivo de disminuir el tamaño del espacio de búsqueda y la dificultad
computacional del problema, se introducen una serie de propiedades estructurales
del \textbf{árbol expansivo mínimo con restricciones de grado} que permiten
reducir el grafo original sin perder optimalidad.

Sea $T^*$ un árbol expansivo mínimo con restricciones de grado del grafo
$G = (V,E)$.

\subsubsection*{Propiedades Estructurales}

\paragraph{Lema 1.}
Sea $v \in V$ un vértice hoja, es decir, un vértice con grado uno en el grafo
original $G$. Entonces, la única arista incidente a $v$ debe pertenecer a $T^*$.

\paragraph{Demostración.}
Dado que $T^*$ es un grafo conexo que contiene todos los vértices de $G$, el vértice
$v$ debe estar conectado al resto del árbol mediante su única arista incidente.
Excluir dicha arista implicaría que $v$ quedaría aislado, contradiciendo la
conectividad de $T^*$. Por tanto, toda arista incidente a un vértice colgante debe
incluirse necesariamente en $T^*$. \hfill $\square$

\vspace{0.2cm}

\paragraph{Lema 2.}
Sea $V_1 = \{ v_i \in V \mid d(v_i) = 1 \}$ el conjunto de vértices cuyo grado máximo
permitido es uno, y sea
\[
E_1 = \{ (v_i,v_j) \in E \mid v_i, v_j \in V_1 \}.
\]
Si $|V| > 2$, entonces ninguna arista de $E_1$ puede pertenecer a $T^*$.

\paragraph{Demostración.}
Supóngase que existe una arista $(v_i,v_j) \in E_1$ incluida en $T^*$. Como
$d(v_i)=d(v_j)=1$, ninguno de estos vértices puede conectarse con ningún otro
vértice adicional en $T^*$. Esto implica que el subgrafo resultante no puede ser
conexo cuando $|V|>2$, lo cual contradice la definición de árbol expansivo.
Por tanto, todas las aristas de $E_1$ deben ser excluidas de $T^*$. \hfill $\square$

\vspace{0.2cm}

\paragraph{Lema 3.}
Sea $v_k$ un vértice de grado dos en $G$, con vecinos $v_i$ y $v_j$. Si no existe
ningún camino entre $v_i$ y $v_j$ que no pase por $v_k$, entonces las aristas
$(v_k,v_i)$ y $(v_k,v_j)$ deben pertenecer a $T^*$.

\paragraph{Demostración.}
Supóngase que alguna de las aristas $(v_k,v_i)$ o $(v_k,v_j)$ no pertenece a $T^*$.
Dado que no existe un camino alternativo entre $v_i$ y $v_j$ que evite el vértice
$v_k$, se deduce que $T^*$ no puede ser conexo, lo cual contradice su definición.
Por tanto, ambas aristas deben incluirse necesariamente en $T^*$. \hfill $\square$

---

\subsubsection*{Algoritmo de Reducción}

A partir de las propiedades anteriores, se define un procedimiento de reducción
que simplifica el grafo original antes de aplicar un algoritmo constructivo.

\begin{lstlisting}
Reduction_DCMST(G = (V,E), d):
    T* <- empty
    1. Eliminar todas las aristas que satisfacen el Lema 2.
    2. Identificar todos los v\'ertices colgantes (grado 1),
       agregar sus aristas incidentes a T* y eliminarlos del grafo.
    3. Identificar v\'ertices que satisfacen el Lema 3,
       agregar las aristas correspondientes a T* y eliminarlas del grafo.
    return G reducido, T*
\end{lstlisting}

El costo computacional de este algoritmo de reducción es:

\begin{itemize}
    \item Paso 1: $O(n^2)$,
    \item Paso 2: $O(n)$,
    \item Paso 3: $O(n^3)$.
\end{itemize}

Por tanto, la complejidad temporal total del algoritmo de reducción es:

\[
O(n^3)
\]

---

\subsubsection*{Construcción del Árbol con Restricciones de Grado}

Una vez reducido el grafo, se construye un árbol expansivo utilizando un algoritmo
greedy basado en el algoritmo clásico de Kruskal, adaptado para respetar las
restricciones de grado.

El algoritmo itera seleccionando aristas de menor costo que conecten componentes
distintas y que no violen los límites de grado de los vértices involucrados. Este
procedimiento continúa hasta obtener $|V|-1$ aristas.


\begin{lstlisting}
MAIN_DCMST(G = (V, E), w, d):
    V* <- V
    E* <- empty
    T* <- (V*, E*)

    1. Ejecutar REDUCTION_DCMST(G, d)
    
    2. E1 <- E

    3. while |E*| < |V| - 1 do
           seleccionar la arista de menor costo
           emin = (vk, vh) en E1
           
           eliminar emin de E1
           
           if vk y vh estan en componentes distintas de T*
              and deg_T*(vk) < d(vk)
              and deg_T*(vh) < d(vh) then
               
               E* <- E* U {emin}
               unir las componentes de vk y vh en T*
           end if
       end while

    4. Aplicar tecnicas de intercambio de aristas
       (1-opt y 2-opt) para mejorar la solucion

    return G* = (V*, E*)
\end{lstlisting}


---

\subsubsection*{Técnicas de Intercambio de Aristas}

Para mejorar la solución obtenida, se emplean técnicas de \emph{edge exchange}.

\paragraph{Intercambio 1-opt.}
Consiste en eliminar una arista del árbol y reemplazarla por una arista externa,
siempre que el resultado sea un árbol. Esta operación puede modificar los grados
de los vértices y, por tanto, debe verificarse nuevamente la restricción de grado.

\paragraph{Intercambio 2-opt.}
Consiste en reemplazar dos aristas del árbol por dos aristas externas, de forma
que el grado de cada vértice se conserve. Esta operación garantiza que la
restricción de grado siga siendo satisfecha y permite mejorar el costo total del
árbol.

---

\subsubsection*{Complejidad del Algoritmo Heurístico}

La complejidad temporal del algoritmo completo es la siguiente:

\begin{itemize}
    \item Reducción del grafo: $O(n^3)$.
    \item Construcción inicial del árbol (Kruskal modificado): $O(n)$.
    \item Intercambio 1-opt: $O(n)$.
    \item Intercambio 2-opt: $O(n^2)$.
\end{itemize}

Por tanto, la complejidad temporal total del algoritmo heurístico es:

\[
\boxed{O(n^3)}
\]

Este enfoque permite obtener soluciones de alta calidad en tiempo polinomial,
haciendo viable el tratamiento de instancias de tamaño considerable.

\newpage
\begin{center}
\LARGE \textbf{Implementación y Análisis Experimental}
\end{center}
\addcontentsline{toc}{section}{Fase 4: Implementación y Análisis Experimental}

El \textbf{Algoritmo Heurístico: Reducción del Árbol de Oportunidades} no garantiza encontrar una solución factible incluso cuando esta existe, 
debido a decisiones greedy irreversibles que pueden saturar restricciones de grado prematuramente. 
Además, las técnicas de edge exchange propuestas solo son aplicables cuando ya se ha construido un 
árbol generador, lo cual no siempre ocurre.


Input donde falla:


\begin{lstlisting}
10 18
4 7 3
8 7 19
3 1 19
4 1 13
6 5 5
9 1 2
3 0 9
2 6 4
0 2 5
5 2 11
8 0 11
6 8 6
2 7 8
7 5 15
3 7 17
2 1 1
1 5 3
7 0 7
1 2 2 3 3 1 3 3 3 2 
\end{lstlisting}


\end{document}
